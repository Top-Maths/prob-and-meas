\documentclass[12pt]{article}

\usepackage{amsmath, amssymb, amsthm}
\usepackage{mathptmx}
\usepackage{datetime}
\title{Probability and Measure \\ Section 1. Borel's Normal Number Theorem}

\date{Last Updated: \today}
\author{Top Maths, mjachi, p1186, cedardrawers}
\newtheorem{theorem}{Theorem}
\newtheorem{lemma}{Lemma}
\newtheorem{corollary}{Corollary}
\newtheorem{definition}{Definition}

\theoremstyle{definition}

\begin{document}
	\maketitle 
	\paragraph{The Unit Interval}
		\begin{definition} ~
			\begin{itemize}
				\item $\Omega = (0, 1]$ denotes the unit interval.
				
				\item $\omega$ denotes a generic point of $\Omega$. 
				
				\item The \textbf{\textit{length}} of an interval $I = (a, b]$ is given by $$|I| = |(a, b]| = b - a.$$ 
				
				\item If $$A = \bigcup_{i = 1}^n I_i = \bigcup_{i = 1}^n (a_i, b_i]$$ where the intervals $I_i$ are disjoint, then assign to $A$ the \textbf{\textit{probability}} $$P(A) = \sum_{i = 1}^n |I_i| = \sum_{i = 1}^n (b_i - a_i).$$
 			\end{itemize}
		\end{definition}
	\paragraph{The Weak Law of Large Numbers}
	
	\paragraph{The Strong Law of Large Numbers}
	
	\paragraph{Strong Law Versus Weak}
	
	\paragraph{Length}
	
	\paragraph{The Measure Theory of Diophantine Approximation}
\end{document}